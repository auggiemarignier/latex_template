%% Below two lines have to be in the top 10 lines of the file. 
%!TEX TS-program = xelatex
%!TEX encoding = UTF-8 Unicode


%%xxxxxxxxxxxxxxxxxxxxxxxxxxxxxxxxxxxxxxxxxxxxxxxxxxxxxxxxxxxxxxxxxxxxxxxxxxxxxxxxxxxxxxxxxxxxxxxxxxxxxxxxxxxxxxxxxx

%% Set in Wiley style.

\RequireXeTeX %Force XeTeX check

%xxx\documentclass[oneside]{article}
%\RequirePackage[oneside]{article}
\LoadClass[oneside]{article}

\RequirePackage{xstring}
\RequirePackage{graphicx}
\RequirePackage[strict]{changepage}
\RequirePackage{etoolbox}
\RequirePackage{textcomp} % for copyright sign
\RequirePackage{lastpage}
\RequirePackage{fancyhdr}
\RequirePackage{enumitem}
\RequirePackage{fancyvrb}  % So can use \begin{Verbatim}. . .\end{Verbatim} if want.
\RequirePackage[x11names]{xcolor}
\RequirePackage{ifthen}
\RequirePackage{listofitems}
\RequirePackage{expl3}
\RequirePackage{trimspaces}
%%%
\AtBeginDocument{\raggedright}
\parindent=0pt
\parskip=.5pc


%% Set journal name:
\def\journalname#1{\def\thejournalname{#1}}
\journalname{Please set Journal Name by using {\tt\string\journalname}}


%xxx REMOVE from non-wiley fork  AND MOVE TO TOP OF CLS FILE <or common file>
%
%
%xx Need BibLaTeX for APA style.  DC
\RequirePackage{apacite}
%\RequirePackage{natbib}
\let\cite\shortcite %xx So get et al. with three authors the first time.
%\let\citep\cite %xx A natbib command.
\let\citet\shortciteA %xx Ditto.
%
\bibliographystyle{apacite}

\RequirePackage{url}


%% Table Footnotes 

\def\tablenotemark#1{\rlap{$^{\rm #1}$}}   %xxxx   

\long\def\tablenotetext#1#2{\vtop{\vskip2pt   %xxxx
\uncentering\noindent\setbox0=\hbox{#1}%
\hskip\saveparindent\ifdim\wd0>1pt
$^{\rm #1}$ \fi{\tablenotefont\ignorespaces #2}\vskip1sp}}

\long\def\tablecomments#1{\vbox{\uncentering  %xxxx
\vskip2pt{\parindent\saveparindent\def\\ {\vskip1sp}\tablenotefont #1}
\vskip 1sp}}

\let\tablecomment\tablecomments  %xxxx  allowing these?
\def\tablenotes{\uncentering\vskip4pt   %xxxx huh?
\tablenotefont\noindent\hskip\saveparindent\ignorespaces}
\def\endtablenotes{\vskip1sp}

\def\uncentering{%   %xxxx
  \let\\\@normalcr
  \rightskip0pt \leftskip0pt
  \parindent\saveparskip \parfillskip0pt plus 1fil\relax}

%xxx  end of REMOVE AND MOVE TO TOP OF CLS FILE  <or...




%\RequirePackage{stringstrings}

\def\articlenumber{\phantom{XX.XX/XX/XXXXEAXXXXXX}}  % Bogus sample number.  Made invisible by request.



% Some definitions used in all styles.
%
\def\authorrunninghead#1{}%\def\theauthorrunninghead{\uppercase{#1}} }%xxx\def\authorrunninghead#1{}} % Called with \initialedfirstauthor in \authors. Disable.
%\def\theauthorrunninghead{\firstauthorinitialednameallcaps}
%\def\theauthorrunninghead#1{\makeinitialedfirstauthor \expandafter\MakeUppercase\expandafter{\firstauthorinitialedname}}
%\def\theauthorrunninghead{\authorrunninghead{}}
%xxx\def\authorrunninghead#1{}} % Called with \initialedfirstauthor in \authors. Disable. %xxx
                           
\def\titlerunninghead#1{}%{\def\therunningheadtitle{#1}}  % Ignored in sample published version.
%
\def\title#1{\def\thetitle{#1}}
\def\authors#1{\def\theauthors{#1} \makeinitialedfirstauthor}%\makeinitialedfirstauthor \authorrunninghead{\firstauthorinitialedname}}
\def\authoraddr#1{\def\theauthoraddr{#1}}  
%
%
\newcounter{affiliationcounter} \setcounter{affiliationcounter}{0}
%
\def\affiliation#1#2{%
    \stepcounter{affiliationcounter} %\theaffiliationcounter (prints counter value)
    \appto{% Add command to \authoraffiliations hook
        %\ifnum\value{affiliationcounter}>1,\fi  %xx
        \theauthoraffiliations}{$^{#1}$#2\ } 
}  

%\def\authoraddress#1{%
%%%xx\typeout{^^J^^J Inside authoraddress ^^J^^J}
%    \def\theauthoraddrress{#1}%
%%%    \findwords[v]{\theauthoraddress}{SI~Corresponding~author:~}
%%%    \ifnum\theresult>0
%%%        \removeword{\theauthoraddress}
%%%        \removeword{\theauthoraddress}
%%%        \removeword{\theauthoraddress}
%%%    \fi
%}
%\let\authoraddr=\authoraddress

%
%
\RequirePackage{marginnote}
%% Put margin on left side only.
%% From: https://tex.stackexchange.com/questions/466512/marginpar-always-on-left-side
%
%\makeatletter
%
%%\usepackage[a4,center,cam,info,]{crop}
%%
%%\setlength{\paperheight}{9truein}%
%%\setlength{\paperwidth}{7truein}%
%%
%%\setlength\textwidth{30pc}
%%\setlength\textheight{40pc}
%%
%%\setlength\marginparwidth{48\p@}%
%%
%%\setlength\oddsidemargin{5pc}
%%\addtolength\oddsidemargin{-1in}    % subtract out the 1 inch driver margin
%%\addtolength\oddsidemargin{\marginparwidth}%
%%\addtolength\oddsidemargin{\marginparsep}%
%
%%\setlength\@tempdima{\paperwidth}
%%\addtolength\@tempdima{-\textwidth}
%%\addtolength\@tempdima{-5pc}
%
%%\setlength\evensidemargin{\@tempdima}
%%\addtolength\evensidemargin{-1in}
%
%\reversemarginpar
%\@mparswitchfalse%
%
%\makeatother
%
%
%
%% Changes to Tufte defs.
%
%
%  Fonts
%
%XX https://tex.stackexchange.com/questions/464406/how-best-to-use-stix2-in-a-document   REMOVE
%XX https://tex.stackexchange.com/questions/43806/what-does-font-name-in-the-fontspec-package-refer-to-on-a-mac
%\usepackage[T1]{fontenc}
\usepackage{amsmath} 
\usepackage{fontspec}
\defaultfontfeatures{Ligatures=TeX, Path=./fonts/,} 
\usepackage{xltxtra}
\usepackage{xunicode}
%\usepackage{stix2}
%\normalfont
\usepackage{unicode-math}
%
%
\setmathfont{STIXTwoMath-Regular}[
Extension={.otf},
Path=./fonts/,
Scale=1]
%
\def\STIXTwo{%
\setmainfont{STIXTwoText}[
Extension={.otf},
Path=./fonts/,
UprightFont={*-Regular},
BoldFont={*-Bold},
ItalicFont={*-Italic},
BoldItalicFont={*-BoldItalic}]
}
\STIXTwo

\setsansfont[Mapping=tex-text,Scale=.9]{FreeSans}%{Helvetica Neue}
\setmonofont[Mapping=tex-text,Scale=1.05]{FreeMono}

%\input ./stix2-otf/STIXTwoText-Regular.otf

%% Set the font sizes and baselines to match Wiley's journal.
%
%% 22/22 \Huge (journal name)
%% 15/16 \LARGE (title)
%% 12/12 \Large (abstract heads)
%% 11/12 \large (section, back matter heads) 
%%
%% 9/12 \normalsize (author list 9/10.8, subsections down to paragraph 9/11)
%% 8/10 \small  (author footnote, tables, table footnotes)
%% 7/9 \footnotesize (margin pars, references )
%%
%% some more missing.  %xxx

% The font sizes
%
\makeatletter
% normalsize
\renewcommand\normalsize{% 9/12
   \@setfontsize\normalsize{9}{12}%
   \abovedisplayskip 8.5pt \@plus3pt \@minus4pt
   \abovedisplayshortskip0pt \@plus2pt
   \belowdisplayshortskip 4pt \@plus2pt \@minus2pt
   \let\@listi\@listI
}   %
\normalbaselineskip=12pt 
\normalsize 

% leading variation on normalsize
\newcommand\authornormalsize{% 9/10.6
   \@setfontsize\normalsize{9}{10.8}%
   \abovedisplayskip 8.5pt \@plus3pt \@minus4pt
   \abovedisplayshortskip0pt \@plus2pt
   \belowdisplayshortskip 4pt \@plus2pt \@minus2pt
   \let\@listi\@listI
}   %


% small
\renewcommand\small{% 8/10
   \@setfontsize\small{8}{10}%%
%   \abovedisplayskip 6pt \@plus2pt \@minus4pt
%   \abovedisplayshortskip 0pt \@pluspt
%   \belowdisplayshortskip 3pt \@pluspt \@minus2pt
%   \def\@listi{\leftmargin\leftmargini
%               \topsep 3pt \@pluspt \@minuspt
%               \parsep 2pt \@pluspt \@minuspt
%               \itemsep \parsep}%
%   \belowdisplayskip \abovedisplayskip
}

% footnotesize
\renewcommand\footnotesize{% 7/9
   \@setfontsize\footnotesize{7}{9}%
%   \abovedisplayskip 5pt \@plus2pt \@minus3pt
%   \abovedisplayshortskip 0pt \@pluspt
%   \belowdisplayshortskip 3pt \@pluspt \@minus2pt
%   \def\@listi{\leftmargin\leftmargini
%               \topsep 3pt \@pluspt \@minuspt
%               \parsep 2pt \@pluspt \@minuspt
%               \itemsep \parsep}%
%   \belowdisplayskip \abovedisplayskip
}

% scriptsize
\renewcommand\scriptsize{\@setfontsize\scriptsize{7}{8}} % 7/8
\newcommand\bibscriptsize{\@setfontsize\scriptsize{7}{9}} % 7/9    % scriptsize with different leading.    %xxx but this is footnotesize
% tiny
\renewcommand\tiny{\@setfontsize\tiny{5}{6}}% 5/6 
% large
\renewcommand\large{\@setfontsize\large{11}{12}} % 11/12
% Large
\renewcommand\Large{\@setfontsize\Large{12}{12}} % 12/12
% LARGE
\renewcommand\LARGE{\@setfontsize\LARGE{14}{16}} % 14/16 
% huge
\renewcommand\huge{\@setfontsize\huge{15}{18}} % 15/18
% Huge
\renewcommand\Huge{\@setfontsize\Huge{22}{22}} % 22/22 

\makeatother


% The fonts using the above sizes
%
%xxx\renewcommand{\normalfont}{\normalsize}%\fontspec{STIXTwoText-Regular}}

\newcommand{\authornormalfont}{\normalfont\authornormalsize\bfseries}%xx \fontspec{STIXTwoText-Bold}}
\newcommand{\titlefont}{\normalfont\huge\bfseries}%fontspec{STIXTwoText-Bold}}
\newcommand{\headerjournalfont}{\normalfont\LARGE\bfseries}
\newcommand{\abstractheadfont}{\normalfont\Large\bfseries}% \fontspec{STIXTwoText-Bold}}
\newcommand{\sectionfont}{\normalfont\large\bfseries}%\fontspec{STIXTwoText-Bold}}
\newcommand{\subsectionfont}{\normalfont\normalsize\bfseries}%\fontspec{STIXTwoText-Bold}}
\newcommand{\authorfootnotesandtablesfont}{\normalfont\small\rmfamily}%\fontspec{STIXTwoText-Regular}}
\newcommand{\marginparreferencesfont}{\normalfont\footnotesize\rmfamily}%\fontspec{STIXTwoText-Regular}}
\newcommand{\marginparreferencesheadsfont}{\normalfont\footnotesize\bfseries}%\fontspec{STIXTwoText-Bold}}
\let\captionfont\marginparreferencesfont
\let\captionlabelfont\marginparreferencesheadsfont
\let\tablenotefont\captionfont
%
\newcommand{\journalfont}{\Huge\fontspec{QTHelvet-Black}}

\gdef\fontblue{SteelBlue4}



%% Set page layout geometry
%
\RequirePackage[%
letterpaper,
%asymmetric,
bindingoffset=0pt,
left=14.5pc,   % 3pc space on left, 2pc on right, so margin text box should be 9.5pc.
right=3pc,
top=4pc,
headsep=1pc,
textwidth=33.5pc,
marginparsep=0pc, % Larger number moves margin text box to left.
marginparwidth=11.5pc,
textheight=48pc,     %xxx 50pc,
headheight=27.65pt, %\baselineskip,
footskip=2pc,%13pt, %10pt,
bottom=4pc,
includehead,
includefoot,
textheight=48pc,     %xxx 50pc,
%xxx lines=20,
%showframe
]{geometry}
%

% For lists.
\setlength\leftmargini   {1pc}
\setlength\leftmarginii  {1pc}
\setlength\leftmarginiii {1pc}
\setlength\leftmarginiv  {1pc}
\setlength\leftmarginv   {1pc}
\setlength\leftmarginvi  {1pc}
\setlength\labelsep      {.5pc}
\setlength\labelwidth    {\leftmargini}
\addtolength\labelwidth{-\labelsep}



% Margin pars
%
% Separation marginpars by 2 lines' worth of space.
\setlength\marginparpush{2\baselineskip}%{10pt}

% A minipage for margin pars, with adjustable vertical size.
%
% #1: minipage height
% #2: minipage contents
\newcommand{\marginminipage}[2]{%
    \vspace*{-20pt}
    %\fbox{%
    %%%%
    \begin{minipage}[t][#1][t]{9.5pc}%  % [vertical, size]{}{horizontal size} 
        #2%
    \end{minipage}
   % }% end fbox
}



% Sections
%
%
% spec fonts, etc.
\makeatletter
\renewcommand\section{\@startsection {section}{1}{\z@}%
                                   {-3.5ex \@plus -1ex \@minus -.2ex}%
                                   {.1ex \@plus.2ex}%
                                   {\sectionfont\color{\fontblue}}}
\renewcommand\subsection{\@startsection{subsection}{2}{\z@}%
                                     {-3.25ex\@plus -1ex \@minus -.2ex}%
                                     {.1ex \@plus .2ex}%
                                     {\subsectionfont \color{\fontblue}}}
\renewcommand\subsubsection{\@startsection{subsubsection}{3}{\z@}%
                                     {-3.25ex\@plus -1ex \@minus -.2ex}%
                                     {.1ex \@plus .2ex}%
                                     {\subsectionfont \color{\fontblue}}}
\renewcommand\paragraph{\@startsection{paragraph}{4}{\z@}%
                                    {1.25ex \@plus1ex \@minus.2ex}%
                                    {-1em}%
                                    {\authornormalfont \color{\fontblue}}}
%\newcommand\subparagraph{\@startsection{subparagraph}{5}{\parindent}%
%                                       {3.25ex \@plus1ex \@minus .2ex}%
%                                       {-1em}%
%                                      {\authornormalsize}}
\makeatother

% Put a period after all section labels, e.g., ``1.2.'' instead of ``1.2''
\makeatletter
\renewcommand\@seccntformat[1]{\csname the#1\endcsname.\ }
\makeatother



%  Bibliography section
%
% Change the bibliography's type size.
\renewcommand\bibliographytypesize{\bibscriptsize}


% Add a margin par to the bibliography's section command.
%
\newcommand{\bibpagemarginpar}{%
    \reversemarginpar\marginpar{%
        \marginminipage{12pt}{%xxx  MAKE number A MACRO
           \raggedright
           \vspace*{9pt}
           \firstpagemarginentry{Acknowledgements}{\textcolor{red}{\textcolor{red}{Your acknowledgements will appear here.}}}            %xxx Need a hook-in.  red
          \vfill
      }
   }
}
%
%
% Add a marginpar to the bibliography.  Using article.cls definition of thebibliography.
\makeatletter
\renewenvironment{thebibliography}[1]
     {\FloatBarrier
      \section*{\refname}%
      \@mkboth{\MakeUppercase\refname}{\MakeUppercase\refname}%
      \bibpagemarginpar%xx  Add a margin par.
      \list{\@biblabel{\@arabic\c@enumiv}}%
           {\settowidth\labelwidth{\@biblabel{#1}}%
            \leftmargin\labelwidth
            \advance\leftmargin\labelsep
              \setlength{\itemsep}{0pt}%xx Reduce the default separation between bib items.  See https://texfaq.org/FAQ-compactbib
            \@openbib@code
            \usecounter{enumiv}%
            \let\p@enumiv\@empty
            \renewcommand\theenumiv{\@arabic\c@enumiv}}%
      \sloppy
      \clubpenalty4000
      \@clubpenalty \clubpenalty
      \widowpenalty4000%
      \sfcode`\.\@m}
     {\def\@noitemerr
       {\@latex@warning{Empty `thebibliography' environment}}%
      \endlist}
\makeatother


% All pages.

% Parse authors list to build running footer.
%
\newcommand{\authorspostfix}{} % If one author, will remain empty; it more than one, will contain ``et al.''

\newcommand{\authorendmarker}{///}


%xxx% See: https://tex.stackexchange.com/questions/150665/trim-spaces-in-a-macro-with-expl3
%\NewDocumentCommand{\trimsurroundingspaces}{m}
% {
%  \tl_trim_spaces:N #1
% }
%\ExplSyntaxOff


% Count the authors and initialize whether to add ``et al.'' to the end of the typeset list.
\newcommand{\countauthors}{%
    \begingroup
        \def\affil{\authorendmarker}
        \def\thanks{\authorendmarker}
        %
	\setsepchar{,/ }% Separate comma list of authors (first level) and names in these authors (second level).
     	\renewcommand{\authorspostfix}{}  % Initialize the postfix.
	%
	\greadlist*\authorslist{\theauthors}% Starred version of \greadlist removes spaces at the beginning and the end of input list.
        %xxx \showitems\authorslist \par %xxx for debugging.
        %
        % Add ``et al.'' to end of authors list if there are more than one.
        \ifthenelse{\authorslistlen > 1} % Adding ``len'' to \authors command gives its length. Very core TeXish construction.
            {
                \gdef\authorspostfix{ et al.} %xxx\typeout{^^J More than one author.^^J}}
             }
            {}
    \endgroup
} % end \countauthors


% Build a version of the first author with leading initials instead of names.
%
% A way of adding to a string that avoids infinate recursion.
\newcommand{\addtostring}[2]{%
%     \makeatletter
%         \g@addto@macro#1{#2}
%     \makeatother
      \expandafter\def\expandafter#1\expandafter{#1{}#2}
%	\edef#1{#1{}#2}
}
%
% Does not take into account unhyphenated double-barreled last names (including  last-name joiners [e.g., ``de'' or `y']; 
%    see https://en.wikipedia.org/wiki/Double-barrelled_name), unreversed Chinese names, preferred initializations/capitalizations, 
%    and probably a lot of other instances.
\newcommand{\makeinitialedfirstauthor}{%
    \begingroup
        \def\affil{\authorendmarker}
        \def\thanks{\authorendmarker}
        %
	%xxx\greadlist\firstauthor{\authorslist[1]}  % Can access each multi-word list element, defined as separated by commas.  Get first word of this.
	%xxx\itemtomacro\authorslist[1]\firstauthor % Can access each multi-word list element, defined as separated by commas.  Get first word of this.
	%
	\countauthors
	%
	\itemtomacro\authorslist[1,1]\firstauthorfirstword
	%xxx first author's first word: \firstauthorfirstword \par %xxx
	%
	\itemtomacro\authorslist[1,2]\firstauthorsecondword
	%xxx first author's second word: \firstauthorsecondword \par  %xxx
	%
	% Rest of first author's name.
	\itemtomacro\authorslist[1]\firstauthorfullnamerest
	%xxx first author's full name, rest of: \firstauthorfullnamerest \par  %xxx
	%
	% Get the substring between the second name and any commands, the latter which presumably come after the last name.
        \def\firstauthorlastname{\StrBetween{\firstauthorfullnamerest}{\firstauthorsecondword}{\authorendmarker}} 	% (xstring)
	%xxxfirst author's last name before trimming spaces: \fbox{\firstauthorlastname} \par  %xxx
	%xx\trimsurroundingspaces{\firstauthorlastname}
	%xxxmakeatletter \trim@spaces@in\firstauthorlastname \makeatother
	%xxx\makeatletter \trim@pre@space@in\firstauthorlastname \makeatother
	%xxx first author's last name after trimming spaces: \fbox{\firstauthorlastname} \par  %xxx
        %
	% Extract first character from the words to be turned into initials. (xstring)
	\def\firstauthorfirstinitial{\StrChar{\firstauthorfirstword}{1}}  % Could have written as: \StrChar{\firstauthorfirstword}{1}[\firstauthorfirstinitial]
	%xxx first author's first initial: \firstauthorfirstinitial \par %xxx
	\def\firstauthorsecondinitial{\StrChar{\firstauthorsecondword}{1}}  % Could have written as: \StrChar{\firstauthorsecondword}{1}[\firstauthorsecondinitial]
	%xxx first author's second initial: \firstauthorsecondinitial \par  %xxx
        %
        	% How many words are in the first author name?
	\def\firstauthornumberofwords{\listlen\authorslist[1]}
	%xxx number of words in first author's name: `\firstauthornumberofwords' \par
	%
	\newcommand{\firstauthorinitialedname}{\firstauthorfirstinitial.} % By default, we turn into an initial the first author's first name.
	%
	% If at least 3 words in author's name, we will produce 2 initials; if 2, will produce only one. 
        \ifthenelse{\numexpr\firstauthornumberofwords > \numexpr 2}
            { 
                 \addtostring{\firstauthorinitialedname}{\firstauthorsecondinitial.} 
            }
            {}
           
        % Add on the last name/rest of the name.
        \addtostring{\firstauthorinitialedname}
        \authorspostfix}
        %xxx first author's name so far: bitchmusic.com  \par
        \def\firstauthorinitialednameallcaps{\expandafter\MakeUppercase\expandafter{\firstauthorinitialedname}}
 }


% Headers and Footers
%
\usepackage{xpatch}

\pagestyle{fancy}
%
\xpretocmd\headrule{\color{\fontblue}}{}{\PatchFailed}
\xpretocmd\footrule{\color{\fontblue}}{}{\PatchFailed}
%
%
\fancyhf{} % clear all header and footer fields
\renewcommand{\headrulewidth}{1pt}  				
\renewcommand{\footrulewidth}{1pt}  
%
% make headers and footers go across the whole page.
% ignore compaints about the `e'.
\fancyheadoffset[loh,reh]{11.5pc}
\fancyfootoffset[loh,reh]{11.5pc}

%
% headers
\def\raiseamount{1pc}
%
\fancyhead[L]{\includegraphics[height=2pc]{./agu-logo-small.pdf}}
\fancyhead[C]{
     \raisebox{\raiseamount}{
         \headerjournalfont  
        \textcolor{\fontblue}{
             \thejournalname
          }
      }
}
\fancyhead[R]{\raisebox{\raiseamount}{\normalfont \textcolor{\fontblue}{\articlenumber}}}
%
% footers
\fancyfoot[L]{\textcolor{\fontblue}{\firstauthorinitialedname}}%allcaps}}
\def\allfooters{%
    \normalfont
    \fancyfoot[L]{\textcolor{\fontblue}{\firstauthorinitialedname}}%allcaps}}
    \fancyfoot[R]{\textcolor{\fontblue}{\thepage\ of \pageref{LastPage}}}
}  
\allfooters
%
%% for the first page only
%\fancypagestyle{firststyle} 
%{
%   \fancyhf{}
%   \allfooters
% }


% Abstract and Plain Language Summary    % Original abstract environment written from scratch in TeX.
%
\def\abstracthead#1{{\abstractheadfont \textcolor{\fontblue}{#1\ }}}
\def\abstractrule{{\color{\fontblue}\rule{\textwidth}{3pt}}}

\newenvironment{plainlanguagesummary}
    {
        \par
        \abstracthead{Plain Language Summary}
        \begingroup
    }
    {
        \par
        \endgroup
    }

\renewenvironment{abstract}
    {
       {\color{\fontblue} \abstractrule}
        \par
        \abstracthead{Abstract}
        \begingroup
    }
    %  Put optional Abstract Body here, including the optional Plain Language Summary environment {plainlanguagesummary}.
    {
        \par
        \endgroup
        \abstractrule
        \vspace*{-18pt}
    }
    
 \newenvironment{unnumbered}  %xxx no example in template file.
     {
         \begin{itemize}[leftmargin=18pt, labelwidth=0pt, labelsep=0pt, itemsep=\baselineskip, topsep=\baselineskip, label={}, listparindent=0pt]
         \begingroup
     }
     {
         \end{itemize}
         \endgroup
     }
 
 
 %xxx no wide figures/tables
 %xxx tell them about no footnotes
 
 
%\renewenvironment{article} {}{}
%
%\else
%\typeout{ELSE not published^^J}   %xxxxxxxxxxxxxxxxxxx
%\fi
%\endinput


% Full-page-width area
%
\def\fullwidthmarginoverhang{-11.5pc}
\def\absfullwidthmarginoverhang{11.5pc}
%
\newlength{\pagewidth}    \setlength{\pagewidth}{\textwidth}    \addtolength{\pagewidth}{\absfullwidthmarginoverhang}
%
\newenvironment{fullwidth}
    {
        % Widen left margin by width of margin text box plus marginpar. 
        \begin{adjustwidth}{\fullwidthmarginoverhang}{}
        \let\oldtextwidth\textwidth
        \let\textwidth\pagewidth
    }   
    {
        \let\textwidth\oldtextwidth
        \end{adjustwidth}
    }
    

%%%
%% Title block
%
% Some more author stuff (defined differently in non-Wiley) for title block.
%%xx Replace \affil stuff with below, from agujournal.cls:    DC
%
\def\affil#1{$^{#1}$\ignorespaces}
\def\thanks#1{\textcolor{red}{(#1)}}   %xxx need to move to affil but how?
%
\renewcommand{\footnote}[1]{%
%    %\ClassError {⟨class-name⟩} {⟨error-text⟩} {⟨help-text⟩}
%    \ClassError {agutexSI2019} 
%    {Your publisher does not allow footnotes.} 
%    {You have to remove all footnotes and integrate them into your main text somehow.\stop}
    \typeout{^^J^^J ERROR: AGU does not permit footnotes outside of the author affiliations.^^J^^J Ignoring.^^J^^J}
    %xxx \show#1
   \makeatletter \@gobble{#1} \makeatother
}

\renewcommand{\titlerunninghead}[1]{%
%    %\ClassError {⟨class-name⟩} {⟨error-text⟩} {⟨help-text⟩}
%    \ClassError {agutexSI2019} 
   \typeout{^^J^^J ERROR: titlerunninghead command is ignored.^^J^^J Ignoring.^^J^^J}
   \makeatletter \@gobble{#1} \makeatother
}


\def\marginparentrysep{1pc}
%
\newcommand{\firstpagemarginentry}[2]{%   
  \marginparreferencesheadsfont \textcolor{\fontblue}{#1:}\\   
  \marginparreferencesfont #2 \\
  \vspace*{\marginparentrysep}
}
%
\renewcommand{\maketitle}{%
      \newpage
      %
      % Suppress running head, not footers.
      \fancypagestyle{plain}{%
          \renewcommand{\headrulewidth}{0pt}%
          \fancyhf{}%
          \allfooters      
      }%
      \thispagestyle{plain}
      
           %
      \setlength{\parskip}{1pc}% space between entries
      %
      \vspace*{-4pc}  % back up top amount								%xx make a macro?
      \vspace*{-1pc}  % back up headsep amount to top of paper
      %
      \vspace*{3pc}   % go down from page top first page amount
      \rightline{\includegraphics[height=2pc]{./agu-logo-large.pdf}}
      \vspace*{2pc}
      %             
      %
      \newcommand{\firstpagemarginpar}{%
            \reversemarginpar\marginpar{%
                 \marginminipage{44.75pc}{%xxx constant?
                       \raggedright
                       \subsectionfont \textcolor{\fontblue}{RESEARCH ARTICLE}\par   
                       \normalfont \articlenumber\par
                       \vspace*{\marginparentrysep}
                      %
                       \firstpagemarginentry{Key Points}{Your list of key points here.}   %xxx hook in?
                       \firstpagemarginentry{Correspondence to}{\theauthoraddr}
                       \firstpagemarginentry{Citation}{Your citation here.}   % Wiley does this.

% Per request, removed everything commented out except \vfill.                      
                         % Received: XX XXX, XXXX\\
                         % Accepted: XX XXX, XXXX\\
                         % \vspace*{\marginparentrysep}
                        
%                       \firstpagemarginentry{Author Contributions}{%
%                           \textcolor{red}{List of %xxx hook in?
%                           author(s) in bold followed by a colon; contribution in regular font.\\
%                           No extra space between entries. One author per line.}
%                       }
                       \vfill  
%                       \marginparreferencesfont 
%                           \begingroup
%                           \color{\fontblue} \textcopyright\ XXXX The Authors. Earth and Space Science published by Wiley Periodicals LLC on behalf of American Geophysical Union.
%                     
%                                This is an open access article under the terms of the Creative Commons Attribution-NonCommercial-NoDerivs License, which permits use and   distribution in any medium, provided the original work is properly cited, the use is non-commercial and no modifications or adaptations are made.
%                          \endgroup
              }   % end marginminipage                                                                                                
          } % end reversemarginpar/marginpar
      } % end firstmarginpar
              
      \gdef\typesetjournal{\begin{fullwidth}  \journalfont \textcolor{\fontblue}{\thejournalname} \end{fullwidth}\par\vspace{1pc}}%xx need \sffamily, too.
      %
      \gdef\typesettitle{%
            \parbox{0.9\textwidth}{%
                  \begingroup
                        \hyphenpenalty 10000\exhyphenpenalty 10000 \raggedright \sloppy \titlefont \textcolor{\fontblue}{\thetitle}
                  \endgroup
            }
            \firstpagemarginpar
            \par
            \vspace*{6pt}
         }
      \gdef\typesetauthors{\authornormalfont\theauthors\par}
      \gdef\typesetauthoraffiliations{\authorfootnotesandtablesfont\theauthoraffiliations}
      %
      \typesetjournal
      \typesettitle
      \typesetauthors
      \typesetauthoraffiliations 
}% end \maketitle



% Floats
%
%\newenvironment{fullwidthfloat}
%    {
%        \begin{fullwidth}   
%           \hspace*{\fullwidthmarginoverhang}
%           \begin{minipage}{\pagewidth}
%    }
%    %
%    {
%            \end{minipage}
%        \end{fullwidth}
%    }

% Watermark.
\RequirePackage[%
angle=270,
fontsize=14pt,
text=This\ paper\ has\ not\ been\ peer-reviewed,
hpos=0.98\paperwidth,
vpos=0.33\paperheight,
hanchor=r,
vanchor=t,
]{draftwatermark}






%%   See Tufte code.
\input{full-width-float} %xxx bring in code

%
%
%
%  end Wiley
%%xxxxxxxxxxxxxxxxxxxxxxxxxxxxxxxxxxxxxxxxxxxxxxxxxxxxxxxxxxxxxxxxxxxxxxxxxxxxxxxxxxxxxxxxxxxxxxxxxxxxxxxxxxxxxxxxxxxxxxxxxxxxxx 






